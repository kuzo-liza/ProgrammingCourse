\documentclass[12pt,a4paper]{report}
\usepackage[utf8]{inputenc}
\usepackage[russian]{babel}
\usepackage[OT1]{fontenc}
\usepackage{amsmath}
\usepackage{amsfonts}
\usepackage{amssymb}
\usepackage{graphicx}
\usepackage{cmap} % поиск в PDF
\usepackage{mathtext} % русские буквы в формулах
%\usepackage{tikz-uml} % uml диаграммы

% TODOs
\usepackage[%
 colorinlistoftodos,
 shadow
]{todonotes}

% Генератор текста
\usepackage{blindtext}

%------------------------------------------------------------------------------

% Подсветка синтаксиса
\usepackage{color}
\usepackage{xcolor}
\usepackage{listings}

 % Цвета для кода
\definecolor{string}{HTML}{B40000} % цвет строк в коде
\definecolor{comment}{HTML}{008000} % цвет комментариев в коде
\definecolor{keyword}{HTML}{1A00FF} % цвет ключевых слов в коде
\definecolor{morecomment}{HTML}{8000FF} % цвет include и других элементов в коде
\definecolor{captiontext}{HTML}{FFFFFF} % цвет текста заголовка в коде
\definecolor{captionbk}{HTML}{999999} % цвет фона заголовка в коде
\definecolor{bk}{HTML}{FFFFFF} % цвет фона в коде
\definecolor{frame}{HTML}{999999} % цвет рамки в коде
\definecolor{brackets}{HTML}{B40000} % цвет скобок в коде

 % Настройки отображения кода
\lstset{
language=C, % Язык кода по умолчанию
morekeywords={*,...}, % если хотите добавить ключевые слова, то добавляйте
 % Цвета
keywordstyle=\color{keyword}\ttfamily\bfseries,
stringstyle=\color{string}\ttfamily,
commentstyle=\color{comment}\ttfamily\itshape,
morecomment=[l][\color{morecomment}]{\#},
 % Настройки отображения
breaklines=true, % Перенос длинных строк
basicstyle=\ttfamily\footnotesize, % Шрифт для отображения кода
backgroundcolor=\color{bk}, % Цвет фона кода
%frame=lrb,xleftmargin=\fboxsep,xrightmargin=-\fboxsep, % Рамка, подогнанная к заголовку
frame=tblr
rulecolor=\color{frame}, % Цвет рамки
tabsize=3, % Размер табуляции в пробелах
showstringspaces=false,
 % Настройка отображения номеров строк. Если не нужно, то удалите весь блок
numbers=left, % Слева отображаются номера строк
stepnumber=1, % Каждую строку нумеровать
numbersep=5pt, % Отступ от кода
numberstyle=\small\color{black}, % Стиль написания номеров строк
 % Для отображения русского языка
extendedchars=true,
literate={Ö}{{\"O}}1
 {Ä}{{\"A}}1
 {Ü}{{\"U}}1
 {ß}{{\ss}}1
 {ü}{{\"u}}1
 {ä}{{\"a}}1
 {ö}{{\"o}}1
 {~}{{\textasciitilde}}1
 {а}{{\selectfont\char224}}1
 {б}{{\selectfont\char225}}1
 {в}{{\selectfont\char226}}1
 {г}{{\selectfont\char227}}1
 {д}{{\selectfont\char228}}1
 {е}{{\selectfont\char229}}1
 {ё}{{\"e}}1
 {ж}{{\selectfont\char230}}1
 {з}{{\selectfont\char231}}1
 {и}{{\selectfont\char232}}1
 {й}{{\selectfont\char233}}1
 {к}{{\selectfont\char234}}1
 {л}{{\selectfont\char235}}1
 {м}{{\selectfont\char236}}1
 {н}{{\selectfont\char237}}1
 {о}{{\selectfont\char238}}1
 {п}{{\selectfont\char239}}1
 {р}{{\selectfont\char240}}1
 {с}{{\selectfont\char241}}1
 {т}{{\selectfont\char242}}1
 {у}{{\selectfont\char243}}1
 {ф}{{\selectfont\char244}}1
 {х}{{\selectfont\char245}}1
 {ц}{{\selectfont\char246}}1
 {ч}{{\selectfont\char247}}1
 {ш}{{\selectfont\char248}}1
 {щ}{{\selectfont\char249}}1
 {ъ}{{\selectfont\char250}}1
 {ы}{{\selectfont\char251}}1
 {ь}{{\selectfont\char252}}1
 {э}{{\selectfont\char253}}1
 {ю}{{\selectfont\char254}}1
 {я}{{\selectfont\char255}}1
 {А}{{\selectfont\char192}}1
 {Б}{{\selectfont\char193}}1
 {В}{{\selectfont\char194}}1
 {Г}{{\selectfont\char195}}1
 {Д}{{\selectfont\char196}}1
 {Е}{{\selectfont\char197}}1
 {Ё}{{\"E}}1
 {Ж}{{\selectfont\char198}}1
 {З}{{\selectfont\char199}}1
 {И}{{\selectfont\char200}}1
 {Й}{{\selectfont\char201}}1
 {К}{{\selectfont\char202}}1
 {Л}{{\selectfont\char203}}1
 {М}{{\selectfont\char204}}1
 {Н}{{\selectfont\char205}}1
 {О}{{\selectfont\char206}}1
 {П}{{\selectfont\char207}}1
 {Р}{{\selectfont\char208}}1
 {С}{{\selectfont\char209}}1
 {Т}{{\selectfont\char210}}1
 {У}{{\selectfont\char211}}1
 {Ф}{{\selectfont\char212}}1
 {Х}{{\selectfont\char213}}1
 {Ц}{{\selectfont\char214}}1
 {Ч}{{\selectfont\char215}}1
 {Ш}{{\selectfont\char216}}1
 {Щ}{{\selectfont\char217}}1
 {Ъ}{{\selectfont\char218}}1
 {Ы}{{\selectfont\char219}}1
 {Ь}{{\selectfont\char220}}1
 {Э}{{\selectfont\char221}}1
 {Ю}{{\selectfont\char222}}1
 {Я}{{\selectfont\char223}}1
 {і}{{\selectfont\char105}}1
 {ї}{{\selectfont\char168}}1
 {є}{{\selectfont\char185}}1
 {ґ}{{\selectfont\char160}}1
 {І}{{\selectfont\char73}}1
 {Ї}{{\selectfont\char136}}1
 {Є}{{\selectfont\char153}}1
 {Ґ}{{\selectfont\char128}}1
 {\{}{{{\color{brackets}\{}}}1 % Цвет скобок {
 {\}}{{{\color{brackets}\}}}}1 % Цвет скобок }
}


%------------------------------------------------------------------------------

\author{Кузовкина.~Е.~О}
\title{Программирование}
\begin{document}
\maketitle
\tableofcontents{}
\chapter{Основные конструкции языка}
%############################################################
\section{Задание 1}
\subsection{Задание 1.1}
Пользователь задает моменты начала и конца некоторого промежутка в часах, минутах, секундах, например, 8:20:35 и 13:15:19. Определить длину промежутка в секундах.
\subsection{Теоретические сведения}

%Конструкции языка, библиотечные функции, инструменты использованные при разработке приложения.
Было использовано:
\begin{enumerate}
\item[1)] функции для ввода и вывода информации (scanf, printf - библиотеки <stdio.h>)

\end{enumerate}

%Сведения о предметной области, которые позволили реализовать алгоритм решения задачи.
Для решения данной задачи нужно было использовать стандартную библиотеку ввода и вывода языка С и элементарные математические операции. Для нахождения промежутка времени необходимо было перевести время начала и конца в секунды и найти их разность.

\subsection{Проектирование}
%Какие функции было решено выделить, какие у этих функций контракты, как организовано взаимодействие с %пользователем (чтение/запись из консоли, из файла, из параметров командной строки), форматы файлов и др.

Было выделено две функции:
\begin{enumerate}
\item[1)]  \verb-a_gap_of_time.c- - для ввода данных из консоли (взаимодействие с пользователем)
\item[2)]  \verb-numbers.c- - для вычисления промежутка времени (бизнес логика)
\end{enumerate}

\subsection{Описание тестового стенда и методики тестирования}
%Среда, компилятор, операционная система, др.
Использовался Qt Creator 3.5.0 (opensource) с GCC 4.9.1 компилятором
Операционная система: Windows 7


%Ручное тестирование, автоматическое, статический анализ кода, динамический.
Использовалось ручное тестирование, автоматическое тестирование не проводилось.
Ошибок и предупреждений не было.

\subsection{Тестовый план и результаты тестирования}

Все тесты были пройдены успешно: полученный результат, совпал с ожидаемым. Результаты тестирования (например,  7 5 3 - 7 часов, 5 минут, 3 секунды):

\begin{enumerate}
\item входные данные - 1 1 1 (начало), 2 2 2 (конец)

ожидаемые значения - 3661

полученые значения - 3661

\item входные данные - 4 4 4 (начало),  4 4 4 (конец)

ожидаемые значения - 0

полученые значения - 0

\end{enumerate}


\subsection{Выводы}

В ходе написания программы не возникло никаких трудностей.

\subsection*{Листинги}
\lstinputlisting
{../sources/final_project/app/first_task.h}
\lstinputlisting
{../sources/final_project/app/first_task.c}
\lstinputlisting
{../sources/final_project/app/a_gap_of_time.c}
\lstinputlisting
{../sources/final_project/app/numbers.c}

%############################################################

\section{Задание 1}
\subsection{Задание 1.2}
Заданы три целых числа: a, b, c. Определить, могут ли они быть длинами сторон треугольника, и если да, является ли является ли данный треугольник остроугольным, прямоугольным или тупоугольным.
\subsection{Теоретические сведения}

%Конструкции языка, библиотечные функции, инструменты использованные при разработке приложения.
Было использовано:
\begin{enumerate}
\item[1)] функции для ввода и вывода информации (scanf, printf - библиотеки <stdio.h>)

\item[2)] конструкция "if"

\end{enumerate}

%Сведения о предметной области, которые позволили реализовать алгоритм решения задачи.
Для решения данной задачи нужно было использовать стандартную библиотеку ввода и вывода языка С и умение определять вид треугольника.
Для этого необходимо было сравнить длину каждой стороны с суммой двух других сторон и сравнить квадрат длины каждой стороны с суммой квадратов двух других сторон.

\subsection{Проектирование}
%Какие функции было решено выделить, какие у этих функций контракты, как организовано взаимодействие с %пользователем (чтение/запись из консоли, из файла, из параметров командной строки), форматы файлов и др.

Было выделено две функции:
\begin{enumerate}
\item[•]   \verb-is_it_a_triangle.c- - для ввода данных из консоли и проверки, существует ли такой треугольник
\item[•]   \verb-what_type_of_triangle_is_it.c- - для вывода типа треугольника
\end{enumerate}

\subsection{Описание тестового стенда и методики тестирования}

%Среда, компилятор, операционная система, др.
Использовался Qt Creator 3.5.0 (opensource) с GCC 4.9.1 компилятором
Операционная система: Windows 7

%Ручное тестирование, автоматическое, статический анализ кода, динамический.
Использовалось ручное тестирование, автоматическое тестирование не проводилось.
Ошибок и предупреждений не было.
\subsection{Тестовый план и результаты тестирования}

Все тесты были пройдены успешно: полученный результат, совпал с ожидаемым. Результаты тестирования:

\begin{enumerate}
\item входные данные - 3 4 5

ожидаемые значения - "Это треугольник. Он правильный"

полученые "This is a right triangle"
\item входные данные - 1 2 3

ожидаемые значения - "Это не треугольник"

полученые "This is not a triangle"
\end{enumerate}


\subsection{Выводы}

При написание данной программы никаких трудностей не было, так как были достаточные знания о геометрических свойствах треугольника.
\subsection*{Листинги}

\lstinputlisting
{../sources/final_project/app/is_it_a_triangle.c}
\lstinputlisting
{../sources/final_project/app/what_type_of_triangle_is_it.c}

%############################################################
\chapter{Циклы}
\section{Задание 2}
\subsection{Задание}

Сформировать расписание звонков в школе. Время начала занятий 9:00. Количество уроков, длительность урока, длительность малого и большого перерыва, число уроков до большого перерыва (он один раз в день) задаются пользователем. Например,
урок    1    2     3     4     5     6
начало 9:00 9:55 10:50 11:45 13:00 13:55
конец 9:45 10:40 11:35 12:30 13:45 14:40

\subsection{Теоретические сведения}

%Конструкции языка, библиотечные функции, инструменты использованные при разработке приложения.
Было использовано:
\begin{enumerate}
\item[1)] функции для ввода и вывода информации (scanf, printf - библиотеки <stdio.h>)

\item[2)] конструкция "if"
\end{enumerate}

%Сведения о предметной области, которые позволили реализовать алгоритм решения задачи.
Для решения данной задачи необходимо понимать структуру школьного расписания.

Было создано расписание с учетом перемен.

\subsection{Проектирование}
%Какие функции было решено выделить, какие у этих функций контракты, как организовано взаимодействие с %пользователем (чтение/запись из консоли, из файла, из параметров командной строки), форматы файлов и др.

Было выделено две функции:
\begin{enumerate}
\item[•] \verb-timetable.c- для ввода данных из консоли (взаимодействие с пользователем)
\item[•] \verb-timetable_work.c- для формирования расписания
\end{enumerate}

\subsection{Описание тестового стенда и методики тестирования}

%Среда, компилятор, операционная система, др.
Использовался Qt Creator 3.5.0 (opensource) с GCC 4.9.1 компилятором
Операционная система: Windows 7


%Ручное тестирование, автоматическое, статический анализ кода, динамический.
Использовалось ручное тестирование, автоматическое тестирование не проводилось.
Ошибок и предупреждений не было.

\subsection{Тестовый план и результаты тестирования}
Все тесты были пройдены успешно: полученный результат, совпал с ожидаемым. Результаты тестирования:

\begin{enumerate}
\item входные данные - 5 40 10 15 3

ожидаемые - 9:00-9:40, 9:50-10:30, 10:40-11:20, 11:35-12:15, 12:25-13:05
полученые - 9:00-9:40, 9:50-10:30, 10:40-11:20, 11:35-12:15, 12:25-13:05

\item входные данные - 4 50 10 20 2

ожидаемые - 9:00-9:50, 10:00-10:50, 11:20-12:10, 12:10-13:00
полученые - 9:00-9:50, 10:00-10:50, 11:20-12:10, 12:10-13:00
\end{enumerate}


\subsection{Выводы}

В ходе написания программы не возникло никаких трудностей.

\subsection*{Листинги}
\lstinputlisting
{../sources/final_project/app/timetable.h}
\lstinputlisting
{../sources/final_project/app/timetable.c}
\lstinputlisting
{../sources/final_project/app/timetable_work.c}

\chapter{Массивы}

\section{Задание 3}

\subsection{Задание}

Матрицу K(m,n) заполнить следующим образом. Элементам, находящимся на периферии (по периметру матрицы), присвоить значение 1, периметру оставшейся подматрицы - значение 2 и так далее до заполнения всей матрицы.

\subsection{Теоретические сведения}

%Конструкции языка, библиотечные функции, инструменты использованные при разработке приложения.
Было использовано:
\begin{enumerate}
\item[1)] функции для ввода и вывода информации (scanf, printf - библиотеки <stdio.h>)
\item[2)] конструкция "if"
\item[3)] функции работы с динамической памятью (malloc - библиотеки <stdlib.h>)
\end{enumerate}

%Сведения о предметной области, которые позволили реализовать алгоритм решения задачи.
Для решения данной задачи необходимо уметь работать с динамической памятью.

Нужно было выделить память, посторить матрицу, заполненную по определенному принципу.

\subsection{Проектирование}
%Какие функции было решено выделить, какие у этих функций контракты, как организовано взаимодействие с %пользователем (чтение/запись из консоли, из файла, из параметров командной строки), форматы файлов и др.

Было выделено две функции:
\begin{enumerate}
\item[1)] \verb-matrix.c- для ввода данных из консоли (взаимодействие с пользователем)
\item[2)] \verb-matrix_maker.c- для формирования матрицы
\end{enumerate}

\subsection{Описание тестового стенда и методики тестирования}
%Среда, компилятор, операционная система, др.
Использовался QtCreator с GCC компилятором
Операционная система: Windows 7


%Ручное тестирование, автоматическое, статический анализ кода, динамический.
Использовалось ручное тестирование, автоматическое тестирование не проводилось.
Ошибок и предупреждений не было.

\subsection{Тестовый план и результаты тестирования}
Все тесты были пройдены успешно: полученный результат, совпал с ожидаемым.

Полученные результаты:
\item входные данные - 3 2

ожидаемые - 1 1 1
            1 1 1
            
полученые - 1 1 1
            1 1 1

\item входные данные - 5 5

ожидаемые - 1 1 1 1 1
            1 2 2 2 1
            1 2 3 2 1
            1 2 2 2 1
            1 1 1 1 1
            
полученые - 1 1 1 1 1
            1 2 2 2 1
            1 2 3 2 1
            1 2 2 2 1
            1 1 1 1 1
 
\item входные данные - 5 5

ожидаемые - 1 1 1 1 1 1
            1 2 2 2 2 1
            1 2 3 3 2 1
            1 2 3 3 2 1
            1 2 2 2 2 1
            1 1 1 1 1 1
            
полученые - 1 1 1 1 1 1
            1 2 2 2 2 1
            1 2 3 3 2 1
            1 2 3 3 2 1
            1 2 2 2 2 1
            1 1 1 1 1 1
            
\subsection{Выводы}

В ходе написания программы не возникло никаких трудностей.

\subsection*{Листинги}
\lstinputlisting
{../sources/final_project/app/matrix.h}
\lstinputlisting
{../sources/final_project/app/matrix.c}
\lstinputlisting
{../sources/final_project/app/matrix_maker.c}

\chapter{Строки}

\section{Задание 4}

\subsection{Задание}

Текст содержит следующие знаки корректуры: \verb-$- - сделать красную строку, \verb-#- - удалить следующее слово, \verb-@- - удалить следующее предложение. Произвести указанную корректировку.
\subsection{Теоретические сведения}
%Конструкции языка, библиотечные функции, инструменты использованные при разработке приложения.
Было использовано:
\begin{enumerate}
\item[1)] функции для ввода и вывода из файла (fopen, fclose- библиотеки <stdio.h>)
\item[2)] конструкция "if"
\item[3)] конструкция "while"
\end{enumerate}
%Сведения о предметной области, которые позволили реализовать алгоритм решения задачи.
Для решения данной задачи необходимо было уметь считывать и записывать информацию в файл, делать корректировку текста.
Нужно было считывать текст из файла, проверять его на наличие определенных символов и записывать исправленный текст в файл.
\subsection{Проектирование}
%Какие функции было решено выделить, какие у этих функций контракты, как организовано взаимодействие с %пользователем (чтение/запись из консоли, из файла, из параметров командной строки), форматы файлов и др.

\begin{enumerate}
The text of this article is in English. \verb-$- See Russian Names in English (Russian Text) for the text in Russian.
 
Different ways of rendering Russian names into English existed in the past, and several standards of transliteration of Cyrillic into English exist now. As a result, there may be several English spelling variants for the same Russian name or surname: Yulia, Yuliya, Julia, Julja (Юлия); Dmitry, Dmitriy, Dmitri, Dimitri (Дмитрий); Yevgeny, Yevgeniy, Evgeny, Evgeni, Evgeniy, Eugeny (Евгений); Tsvetaeva, Tsvetayeva, Cvetaeva (Цветаева); Zhukovsky, Zhukovski, Zhukovskiy, Jukovsky (Жуковский).
 
In some cases, the number \verb-$- of existing English \verb-$- variants is really intimidating. For example, # Муравьёв, a common Russian last name, is represented by more than fifteen variants of spelling in English: Muravyov, Myravyev, Muraviev, Muraviov, Murav'ev, Muravev, Murav'yev, Murav'ov, Muravjov, Muravjev, Mouravieff, Muravieff, Mouravief, Muravief, Muraviof, Muravioff.
 
Generally, # transliteration of Russian names into Latin is English-oriented now. @ But many Russian names were transliterated according to the French language in the past, and transliteration on the basis of French was the norm for names and surnames in our travel passports until recently. As a result, French-oriented transliteration variants of Russian names are still common. Also, English spelling of Russian names is influenced by tradition and people's personal preferences.
 
Modern ways of rendering Russian names into English \verb-$- try to preserve, as much as possible, both the pronunciation and the recognizable written look of the original Russian name. This material offers examples of typical English spelling variants for Russian names and linguistic recommendations for rendering Russian names into English. Bear in mind that your name should be written in the same way in all of your travel documents because discrepancies may lead to problems when travelling.
\end{enumerate}
\subsection{Описание тестового стенда и методики тестирования}
%Среда, компилятор, операционная система, др.
Использовался QtCreator с GCC компилятором
Операционная система: Windows 7
%Ручное тестирование, автоматическое, статический анализ кода, динамический.
Использовалось ручное тестирование, автоматическое тестирование не проводилось.
Ошибок и предупреждений не было.
\subsection{Тестовый план и результаты тестирования}
Все тесты были пройдены успешно: полученный результат, совпал с ожидаемым, то есть была произведена корректировка текста.

Полученный текст:
The text of this article is in English. 
   See Russian Names in English (Russian Text) for the text in Russian.
 
Different ways of rendering Russian names into English existed in the past, and several standards of transliteration of Cyrillic into English exist now. As a result, there may be several English spelling variants for the same Russian name or surname: Yulia, Yuliya, Julia, Julja (Юлия); Dmitry, Dmitriy, Dmitri, Dimitri (Дмитрий); Yevgeny, Yevgeniy, Evgeny, Evgeni, Evgeniy, Eugeny (Евгений); Tsvetaeva, Tsvetayeva, Cvetaeva (Цветаева); Zhukovsky, Zhukovski, Zhukovskiy, Jukovsky (Жуковский).
 
In some cases, the number 
   of existing English 
   variants is really intimidating. For example,  Муравьёв, a common Russian last name, is represented by more than fifteen variants of spelling in English: Muravyov, Myravyev, Muraviev, Muraviov, Murav'ev, Muravev, Murav'yev, Murav'ov, Muravjov, Muravjev, Mouravieff, Muravieff, Mouravief, Muravief, Muraviof, Muravioff.
 
Generally,  transliteration of Russian names into Latin is English-oriented now. . As a result, French-oriented transliteration variants of Russian names are still common. Also, English spelling of Russian names is influenced by tradition and people's personal preferences.
 
Modern ways of rendering Russian names into English 
   try to preserve, as much as possible, both the pronunciation and the recognizable written look of the original Russian name. This material offers examples of typical English spelling variants for Russian names and linguistic recommendations for rendering Russian names into English. Bear in mind that your name should be written in the same way in all of your travel documents because discrepancies may lead to problems when travelling.
\subsection{Выводы}
В ходе написания программы не возникло никаких трудностей.
\subsection*{Листинги}
\lstinputlisting
{../sources/final_project/app/strings.c}
\chapter{Задание на классы}
\section{Множество}
\subsection{Задание 1}
Множество
Реализовать класс МНОЖЕСТВО (целых чисел). Требуемые методы: конструктор, деструктор, копирование, сложение множеств, пересечение множеств, добавление в множество, включение в множество.
\subsection{Теоретические сведения}
Множество - совокупность некоторых объектов (элементов). 

Объединение множеств - образование нового множества содержащего все элементы обоих множеств, без повторений элементов. 

Равные множества - множества, содержащие одинаковые элементы. 

Пересечение множеств - образуется множество, содержащее только элементы принадлежащие обоим множествам. 

Подмножество множества - часть множества.

Разность множеств - множество, содержащее элементы одного множества, но не содержащее элементы другого.
%Конструкции языка, библиотечные функции, инструменты использованные при разработке приложения.
Было использовано:
\begin{enumerate}
\item[1)] потоки ввода и вывода информации  - (iostream, ostream - библиоте-
ки <stdlib>)
\item[2)] конструкция "if"
\item[3)] циклы "while" и "for"
\item[4)] классы
\end{enumerate}
%Сведения о предметной области, которые позволили реализовать алгоритм решения задачи.
Для решения данной задачи необходимо уметь работать с классами и потоками ввода и вывода.
Нужно было создать класс и его методы.
\subsection{Проектирование}
%Какие функции было решено выделить, какие у этих функций контракты, как организовано взаимодействие с %пользователем (чтение/запись из консоли, из файла, из параметров командной строки), форматы файлов и др.
В классе были реализованы следующие поля: 
\item[1)] mas - массив, представляющий из себя множество
\item[2)] k - переменная, использующаяся в качестве счетчика

и следующие методы:
\item[1)] Add - функция, которая добавляет элементы в множество
\item[2)] Sub - функция, которая убирает элементы из множества
\item[3)] HowMany - функция, которая возвращает количество элементов в множестве

Также были перегружены следующие операторы:
\item[1)] "+=" - оператор, использующий функцию Add, добавляет элемент в множество или объединяет два множества
\item[2)] "-=" - оператор, использующий функцию Sub, убирает элемент из множества или вычитает одно множество из другого множества
\item[3)] "*=" - пересечение двух множеств
\item[4)] "==" - логический оператор, проверяющий равны ли два множества
\item[5)] "<=" - логический оператор, проверяющий является ли одно множество подмножеством другого
\item[6)] "<<", ">>" - ввод и вывод в поток

\subsection{Описание тестового стенда и методики тестирования}
%Среда, компилятор, операционная система, др.
Использовался QtCreator с GCC компилятором
Операционная система: Windows 7
%Ручное тестирование, автоматическое, статический анализ кода, динамический.
Использовалось ручное тестирование, автоматическое тестирование
не проводилось. Ошибок и предупреждений не было.
\subsection{Тестовый план и результаты тестирования}
Все тесты были пройдены успешно: полученный результат, совпал с ожидаемым.
\subsection{Выводы}
В ходе написания программы не возникло никаких трудностей
\subsection*{Листинги}
\lstinputlisting
{../sources/final_project/plurality/plurality.cpp}
\end{document} 